\section{Obiettivo}
L'obiettivo di questo progetto è quello di realizzare un programma scritto in \textbf{OCaml} che risolva il \textit{problema} della \textit{colorazione di un grafo}: riuscire a colorare, se possibile, ogni nodo del grafo in modo da non avere mai nodi adiacenti con lo stesso colore. In modo più formale, dato un grafo \lstinline[style=cmd]|g| ed un numero massimo di colori utilizzabili \lstinline[style=cmd]|N|, assegnare un colore ( da $0$ a $N-1$) ai nodi in modo tale che non esistano nodi adiacenti con lo stesso colore; qualora non sia possibile, riportare un errore.
 
\section{Struttura del Progetto}
%TODO: aggiungere la struttura del progetto. forse va bene una foto o il risultato di un LS.

\section{Codice}
%TODO: spiegare prima come vengono rappresentati i colori (tipo il fatto che -1 vuol dire non colorato). E spiegare brevemente alcune funzioni importanti, non tutte ed in modo veloce e conciso.

\section{Dimostrazione}
%TODO: aggiungere foto e mostrare come funziona in pratica il progetto.

		
		\dirtree{%
			.1 .
			.1 Makefile.
			.1 src.
			.2 main.ml.
			.2 main.mli.
			.2 graphUtils.ml.
			.2 graphUtils.mli.
			.2 data.ml.
			.2 data.mli.
			.2 printer.ml.
			.2 printer.mli.
			.2 rappresentazione\_grafo.
			.3 rappresentazione\_grafo.py.
		}




